\documentclass[a4paper,10pt,headings=standardclasses]{scrartcl}

\usepackage{style}

\titlehead{Jendrik Stelzner \hfill \texttt{jendrikstelzner.de/seminars/dg\_hopf.pdf} \hfill May 20, 2019}
% \subject{Graduate Seminar on Topology}
\title{Differential Graded Hopf Algebras~I}
% \footnote{Available online at \texttt{jendrikstelzner.de/dg\_hopf\_extended.pdf}.}}
% \subtitle{Introducing Signs}
\author{}
\date{}





\begin{document}

\maketitle

\vspace{-4em}

In the following~$k$ denotes a field.
All vector spaces, algebras, tensor products, etc.\ are over~$k$, unless otherwise stated.
All occuring maps are linear unless otherwise stated.
Additional constraints on~$\ringchar(k)$ are made explicit when used.





\section{Preliminary Notions and Notations}

A \defemph{graded vector space} is a vector space~$V$ together with a \defemph{grading}~$V = \bigoplus_{n \in \Integer} V_n$.
The elements~$v \in V_n$ are \defemph{homogeneous} of \defemph{degree}~$\hdeg{v} = n$.
\begin{center}
  \fbox{Whenever we write~$\hdeg{v}$ the element~$v$ is assumed to be homogeneous.}
\end{center}
A map~$f \colon V \to W$ between graded vector spaces is \defemph{graded} of \defemph{degree}~$d = \hdeg{f}$ if~$f(V_n) \subseteq V_{n+d}$ for all~$n$.
A \defemph{differential} on~$V$ is a map~$V \to V$ of degree~$-1$ with~$d^2 = 0$.
A \defemph{\dgv} is a graded vector space together with a differential, i.e.\ a chain complex;
the usual definitions and assertions about chain complexes apply.
A \defemph{\dgsub} is a chain subcomplex.
We always regard graded objects as differential graded objects with zero differential.
\begin{center}
  \fbox{graded $\longleftrightarrow$ differential graded with~$d = 0$}
\end{center}
% Hence every statement about {\dgos} entails a statement about graded objects.e
If~$V$,~$W$ are graded vector spaces then~$V \tensor W$ is also one with~$\hdeg{v \tensor w} = \hdeg{v} + \hdeg{w}$, i.e.~$(V \tensor W)_n = \bigoplus_{i+j=n} V_i \tensor W_j$.
The \defemph{twist map}~$\tau \colon V \tensor W \to W \tensor V$ is given by
\[
  \tau(v \tensor w)
  =
  (-1)^{\hdeg{v} \hdeg{w}}
  w \tensor v \,.
\]
We hence adhere to the Koszul-Quillen \defemph{sign convention}:
\begin{center}
  \fbox{Whenever homogeneous~$x$,~$y$ are swapped the sign~$(-1)^{\hdeg{x} \hdeg{y}}$ is introduced.}
\end{center}
If~$V$,~$W$ are {\dgvs} then~$\dgHom(V,W)$ is the {\dgv} with
\begin{align*}
  \dgHom(V,W)_n
  &=
  \{
    \text{graded maps~$V \to W$ of degree~$n$}
  \} \,,
  \\
  d(f)
  &=
  d \circ f - (-1)^{\hdeg{f}} f \circ d \,.
\end{align*}
% (The spaces~$\dgHom(V,W)_n$ are linearly independent in~$\Hom_k(V,W)$ whence we can regard~$\dgHom(V,W)$ as a linear subspace of~$\Hom_k(V,W)$.)
If~$f \colon V \to V'$,~$g \colon W \to W'$ are graded maps then~$f \tensor g \colon V \tensor V' \to W \tensor W'$ is given by
\[
  (f \tensor g)(v \tensor w)
  =
  (-1)^{\hdeg{g} \hdeg{v}}
  f(v) \tensor g(w) \,;
\]
in particular~$\hdeg{f \tensor g} = \hdeg{f} + \hdeg{g}$.
If~$V$,~$W$ are {\dgvs} then~$V \tensor W$ is a {\dgv} with~$d_{V \tensor W} = d_V \tensor {\id} + {\id} \tensor d_W$;
more explicitely,
\[
  d(v \tensor w)
  =
  d(v) \tensor w + (-1)^{\hdeg{v}} v \tensor d(w) \,.
\]
Higher tensor products are defined inductively.
The twist map~$\tau$ is an isomorphism of {\dgvs}.%
\footnote{The naive twist map~$v \tensor w \mapsto w \tensor v$ is not a morphism of {\dgvs}.}
We regard~$k$ as a~{\dgv} concentrated in degree~$0$.
% Then the natural isomorphism~$k \tensor V \cong V$ and~$V \tensor k \cong V$ are isomorphism of~{\dgvs}.
% It holds that~$\cycles(V \tensor W) = \cycles(V) \tensor \cycles(W)$ as graded vector spaces, and the \defemph{algebraic Künneth isomorphism} is the natural isomorphism of graded vector spaces
% \[
%   \homology(V \tensor W)
%   \cong
%   \homology(V) \tensor \homology(W) \,,
%   \quad
%   [v \tensor w]
%   \mapsfrom
%   [v] \tensor [w] \,.
% \]







\section{Differential Graded Algebras}

\begin{definition}
  A \defemph{differential graded algebra} or \defemph{\dga} is a {\dgv}~$A$ together with morphisms of {\dgvs}~$m \colon A \tensor A \to A$ and~$u \colon k \to A$ that make the algebra diagrams
  \[
    \begin{tikzcd}[column sep = large]
      A \tensor A \tensor A
      \arrow{r}[above]{{\id} \tensor m}
      \arrow{d}[left]{m \tensor {\id}}
      &
      A \tensor A
      \arrow{d}[right]{m}
      \\
      A \tensor A
      \arrow{r}[below]{m}
      &
      A
    \end{tikzcd}
    \qquad
    \begin{tikzcd}
      k \tensor A
      \arrow{d}[left]{u \tensor {\id}}
      &
      A
      \arrow{l}[above]{\sim}
      \arrow[equal]{d}
      \arrow{r}[above]{\sim}
      &
      A \tensor k
      \arrow{d}[right]{{\id} \tensor u}
      \\
      A \tensor A
      \arrow{r}[below]{m}
      &
      A
      &
      A \tensor A
      \arrow{l}[below]{m}
    \end{tikzcd}
  \]
  commute.
  The {\dga}~$A$ is \defemph{graded commutative} if the diagram
  \[
    \begin{tikzcd}[column sep = small]
      A \tensor A
      \arrow{rr}[above]{\tau}
      \arrow{dr}[below left]{m}
      &
      {}
      &
      A \tensor A
      \arrow{dl}[below right]{m}
      \\
      {}
      &
      A
      &
      {}
    \end{tikzcd}
  \]
  commutes.
  A \defemph{morphism} of {\dgas}~$f \colon A \to B$ is a morphism of {\dgvs} such that the following diagrams commute:
  \[
    \begin{tikzcd}[column sep = large]
      A \tensor A
      \arrow{r}[above]{f \tensor f}
      \arrow{d}[left]{m}
      &
      B \tensor B
      \arrow{d}[right]{m}
      \\
      A
      \arrow{r}[above]{f}
      &
      B
    \end{tikzcd}
    \qquad
    \begin{tikzcd}[column sep = small]
      {}
      &
      k
      \arrow{dl}[above left]{u}
      \arrow{dr}[above right]{u}
      &
      {}
      \\
      A
      \arrow{rr}[above]{f}
      &
      {}
      &
      B
    \end{tikzcd}
  \]
\end{definition}

\begin{definition}
  A graded map~$\delta \colon A \to A$ for a graded algebra~$A$ is a \defemph{derivation} if
  \begin{align*}
    \delta \circ m
    &=
    m \circ (\delta \tensor {\id} + {\id} \tensor \delta) \,;
  \shortintertext{more explicitely,}
    \delta(ab)
    &=
    \delta(a) b + (-1)^{\hdeg{\delta} \hdeg{a}} a \delta(b) \,.
  \end{align*}
\end{definition}

\begin{remark}
  \leavevmode
  \begin{enumerate}
    \item
      A {\dga} is the same as a graded algebra~$A$ (in particular~$\hdeg{1} = 0$) together with a differential~$d$ such that~$d(1) = 0$ and
      \[
        d(a \cdot b)
        =
        d(a) \cdot b + (-1)^{\hdeg{a}} a \cdot d(b) \,,
      \]
      i.e.\ such that~$d$ is a graded derivation (of degree~$-1$).
%       \[
%         \begin{tikzcd}[column sep = 8em, row sep = large]
%           \begin{tabular}{c} graded \\ vector spaces \end{tabular}
%           \arrow{r}[above]{\text{multiplication}}
%           \arrow{d}[left]{\text{differential}}
%           &
%           \begin{tabular}{c} graded \\ algebras \end{tabular}
%           \arrow{d}[right]{\text{differential}}
%           \\
%           \text{\dgvs}
%           \arrow{r}[below]{\text{multiplication}}
%           &
%           \text{\dgas}
%         \end{tikzcd}
%       \]
    \item
      The graded commutativity of~$A$ means~$ab = (-1)^{\hdeg{a} \hdeg{b}} ba$.
      If~$\hdeg{a}$ or~$\hdeg{b}$ is even then~$ab = ba$;
      if~$\hdeg{a}$ is odd and~$\ringchar(k) \neq 2$ then~$a^2 = 0$.
    \item
      A morphism~$f$ of~{\dgas} is the same as a morphism of the underlying graded algebras that commutes with the differentials.
      (No additional signs occur since~$\hdeg{f} = 0$.)
  \end{enumerate}
\end{remark}

\begin{examples}
  \leavevmode
  \begin{enumerate}
    \item
      Every algebra~$A$ is a {\dga} concentrated in degree~$0$, in particular~$A = k$.
    \item
      For any~{\dgv}~$V$ the algebra structure of~$\End_k(V)$ restricts to a {\dga} structure on~$\dgEnd(V) = \dgHom(V,V)$.
    \item
      If~$V$ is a~{\dgv} then~$\Tensor(V) = \bigoplus_{d \geq 0} V^{\tensor d}$ is again a {\dgv} with
      \begin{align*}
        \hdeg{v_1 \dotsm v_n}
        &=
        \hdeg{v_1} + \dotsb + \hdeg{v_n} \,,
      \\
        d(v_1 \dotsm v_n)
        &=
        \sum_{i=1}^n
        (-1)^{\hdeg{v_1} + \dotsb + \hdeg{v_{i-1}}}
        v_1 \dotsm d(v_i) \dotsm v_n \,.
      \end{align*}
      This makes~$\Tensor(V)$ into a {\dga}, with multiplication given by concatination
      \[
        (v_1 \dotsm v_n) \cdot (v_{n+1} \dotsm v_m)
        =
        v_1 \dotsm v_m \,.
      \]

      The inclusion~$V \to \dgTensor(V)$ is a morphism of {\dgvs} and if~$f \colon V \to A$ is any morphism of {\dgvs} into a {\dga}~$A$ then~$f$ extends uniquely to a morphism of {\dgas}~$F \colon \dgTensor(V) \to A$:
      \[
        \begin{tikzcd}
          \dgTensor(V)
          \arrow[dashed]{r}[above]{F}
          &
          A
          \\
          V
          \arrow{u}
          \arrow{ur}[below right]{f}
          &
          {}
        \end{tikzcd}
      \]
      The {\dga}~$\dgTensor(V)$ is the \defemph{differential graded tensor algebra} on~$V$.
%   TODO: {Add the shuffle dg algebra.}
%     \item
%       If~$V$ is any vector space then the symmetric algebra~$\Symm(V)$ is a graded algebra and a commutative algebra, but not a graded commutative algebra.
%       The exterior algebra~$\Exterior(V)$ is a graded algebra, it is in general not a commutative algebra (unless~$\dim V \leq 1$), but it is a graded commutative algebra.  
  \end{enumerate}
\end{examples}

\begin{lemma}
  Let~$A$,~$B$ be {\dgas}.
  \begin{enumerate}
    \item
      The tensor product $A \tensor B$ becomes a {\dga} with
      \begin{gather*}
        m_{A \tensor B}
        \colon
        A \tensor B \tensor A \tensor B
        \xlongto{{\id} \tensor \tau \tensor {\id}}
        A \tensor A \tensor B \tensor B
        \xlongto{m \tensor m}
        A \tensor B
      \\
        u_{A \tensor B}
        \colon
        k
        \xlongto{\sim}
        k \tensor k
        \xlongto{u \tensor u}
        A \tensor B \,.
      \end{gather*}
      More explicitely,~$1_{A \tensor B} = 1_A \tensor 1_B$ and~$(a_1 \tensor b_1) (a_2 \tensor b_2) = (-1)^{\hdeg{a_2} \hdeg{b_1}} a_1 a_2 \tensor b_1 b_2$.
    \item
      If~$f \colon A \to A'$ and~$g \colon B \to B'$ are morphism of {\dgas} then so is~$f \tensor g$.
    \item
      The twist map~$\tau \colon A \tensor B \to B \tensor A$ is an isomorphism of {\dgas}.
    \item
      The {\dga}~$A^{\op}$ is given by~$u_{A^{\op}} = u_A$ and~$m^{\op} = m_A \circ \tau$.
      If~$\cdot$ denotes the multiplication in~$A$ and~$*$ the multiplication in~$A^{\op}$ then more explicitely
      \[
        1_A
        =
        1_{A^\op} \,,
        \qquad
        a * b
        =
        (-1)^{\hdeg{a} \hdeg{b}} b \cdot a \,.
        \tag*{\qed} %dirty hack
      \]
  \end{enumerate}
\end{lemma}

\begin{warning}
  If~$A \tensor_k B$ is the non-dg tensor product with~$(a \tensor b)(a' \tensor b') = a a' \tensor b b'$ then~$A \tensor B \neq A \tensor_k B$ as algebras, i.e.\ the underlying algebra of~$A \tensor B$ is not the tensor product of the underlying algebras of~$A$ and~$B$.
  The underlying algebra of~$A^{\op}$ is similarly not the opposite of the underlying algebra of~$A$.
\end{warning}

\begin{definition}
  A \defemph{\dgi} in a {\dga}~$A$ is a {\dgsub} that is also an ideal.%
  \footnote{By an ideal we always mean a two-sided ideal.}
\end{definition}

\begin{lemma}
  If~$I$ is a~{\dgi} in~$A$ then~$A/I$ inherits the structure of a {\dga}.
  \qed
\end{lemma}

\begin{lemma}
  An ideal~$I$ is a {\dga}~$A$ is a {\dgi} if and only if~$I$ is generated by homogeneous elements~$x_\alpha$ with~$d(x_\alpha) \in I$ for every~$\alpha$.
  (Being a {\dgi} can be checked on homegenous generators.)
\end{lemma}

\begin{proof}
  That~$I$ is a graded ideal if and only if it is generated by homogeneous elements is  well-known, see \cite[IX, 2.5]{lang} or \cite[II.{\S}11.3]{bourbaki}.
  It remains to show that~$d(I) \subseteq I$ if~$d(x_\alpha) \in I$ for every~$\alpha$:
  The ideal~$I$ is spanned by~$a x_\alpha b$ with~$a, b \in A$ homogeneous, and
  \[
    d(a x_\alpha b)
    =
      d(a) x_\alpha b
    + (-1)^{\hdeg{a}} a d(x_\alpha) b
    + (-1)^{\hdeg{a} + \hdeg{x_\alpha}} a x_\alpha d(b)
    \in
    I
  \]
  since~$x_\alpha, d(x_\alpha) \in I$.
\end{proof}

\begin{definition}
  The \defemph{{\dgcom}} in a {\dga}~$A$ is the bilinear extension of
  \[
    [a,b]
    \defined
    ab - (-1)^{\hdeg{a}\hdeg{b}} ba \,.
  \]
\end{definition}

\begin{example}
  \label{dg symmetric algebra}
  Let~$V$ be a {\dgv}.
  The ideal
  \[
    I
    \defined
    \bigl(
      [v,w]
    \suchthat[\big]
      \text{$v,w \in V$ are homogeneous}
    \bigr)
  \]
  is a {\dgi} in~$\dgTensor(V)$ since the generators~$[v,w]$ are homogeneous with
  \[
    d([v,w])
    =
    [d(v), w] + (-1)^{\hdeg{v}} [v, d(w)]
    \in
    I \,.
  \]
  The {\dga}~$\dgSymm(V) \defined \dgTensor(V)/I$ is the \defemph{differential graded symmetric algebra} on~$V$.
  If~$S$ is any other graded symmetric {\dga} and~$f \colon V \to S$ any morphism of {\dgv} then~$f$ extends uniquely to a morphism of~{\dgas}~$F \colon \dgSymm(V) \to S$:
  \[
    \begin{tikzcd}
      \dgSymm(V)
      \arrow[dashed]{r}[above]{F}
      &
      S
      \\
      V
      \arrow{u}
      \arrow{ur}[below right]{f}
      &
      {}
    \end{tikzcd}
  \]

\end{example}

% TODO: Fix this; probably not needed.
% \begin{remark}
%   Let~$V, W$ be {\dgvs}.
%   \begin{enumerate}
%     \item
%       The inclusions~$V, W \to V \oplus W$ induce morphisms of {\dghs}
%       \[
%         \dgSymm(V), \dgSymm(W)
%         \to
%         \dgSymm(V \oplus W)
%       \]
%       which give an isomorphism of {\dghs}~$\dgSymm(V) \tensor \dgSymm(W) \to \dgSymm(V \oplus W)$.
% %     TODO: Check this.
%   \end{enumerate}
%   Let now~$\ringchar(k) \neq 2$.
%   \begin{enumerate}[resume]
%     \item
%       If~$V$ is concentrated in even degrees then~$\dgSymm(V) = \Symm(V)$ and if~$V$ is concentrated in odd degrees then~$\dgSymm(V) = \Exterior(V)$, both with the gradings induced by~$V$.
%     \item
%       If~$V_{\text{even}} = \bigoplus_{n \in \Integer} V_{2n}$ and~$V_{\text{odd}} = \bigoplus_{n \in \Integer} V_{2n+1}$ then
%       \[
%         \dgSymm(V)
%         =
%         \dgSymm(V_{\text{even}} \oplus V_{\text{odd}})
%         \cong
%         \dgSymm(V_{\text{even}}) \tensor \dgSymm(V_{\text{odd}})
%         \cong
%         \Symm(V_{\text{even}}) \tensor \Exterior(V_{\text{odd}}) \,.
%       \]
%       Moreover, the tensor factors~$\Symm(V_{\text{even}})$ and~$\Exterior(V_{\text{odd}})$ strictly commute in~$\dgSymm(V)$ (i.e.~$xy = yx$) so that
%       \[
%         \dgSymm(V)
%         \cong
%         \Symm(V_{\text{even}}) \tensor_k \Exterior(V_{\text{odd}})
%       \]
%       as algebras, where~$\tensor_k$ denotes \enquote{tensor product without signs}.
%   \end{enumerate}
% \end{remark}

\begin{proposition}
  If~$A$ is a {\dga} then~$\cycles(A)$ is a graded subalgebra of~$A$,~$\boundaries(A)$  is a graded ideal in~$\cycles(A)$ and~$\homology(A)$ is hence a graded algebra.
  \qed
\end{proposition}




\section{Differential Graded Coalgebras}

\begin{definition}
  A \defemph{differential graded coalgebra} or \defemph{\dgc} is a {\dgv}~$C$ together with morphisms of~{\dgv}~$\Delta \colon C \to C \tensor C$ and~$\varepsilon \colon C \to k$ that make the diagrams
  \[
    \begin{tikzcd}[column sep = large]
      C
      \arrow{r}[above]{\Delta}
      \arrow{d}[left]{\Delta}
      &
      C \tensor C
      \arrow{d}[right]{{\id} \tensor \Delta}
      \\
      C \tensor C
      \arrow{r}[below]{\Delta \tensor {\id}}
      &
      C \tensor C \tensor C
    \end{tikzcd}
    \qquad
    \begin{tikzcd}
      C \tensor C
      \arrow{d}[left]{\varepsilon \tensor {\id}}
      &
      C
      \arrow{l}[above]{\Delta}
      \arrow[equal]{d}
      \arrow{r}[above]{\Delta}
      &
      C \tensor C
      \arrow{d}[right]{\varepsilon \tensor {\id}}
      \\
      k \tensor C
      \arrow{r}[above]{\sim}
      &
      C
      &
      C \tensor k
      \arrow{l}[above]{\sim}
    \end{tikzcd}
  \]
  commute.
  The {\dgc}~$C$ is \defemph{graded cocommutative} if the diagram
  \[
    \begin{tikzcd}[column sep = small]
      {}
      &
      C
      \arrow{dl}[above left]{\Delta}
      \arrow{dr}[above right]{\Delta}
      &
      {}
      \\
      C \tensor C
      \arrow{rr}[above]{\tau}
      &
      {}
      &
      C \tensor C
    \end{tikzcd}
  \]
  commutes.
  A \defemph{morphism} of {\dgc}~$f \colon C \to D$ is a morphism of {\dgvs} such that the following diagrams commute:
  \[
    \begin{tikzcd}[column sep = large]
      C
      \arrow{r}[above]{f}
      \arrow{d}[left]{\Delta}
      &
      D
      \arrow{d}[right]{\Delta}
      \\
      C \tensor C
      \arrow{r}[above]{f \tensor f}
      &
      D \tensor D
    \end{tikzcd}
    \qquad
    \begin{tikzcd}[column sep = small]
      C
      \arrow{rr}[above]{f}
      \arrow{dr}[below left]{\varepsilon}
      &
      {}
      &
      D
      \arrow{dl}[below right]{\varepsilon}
      \\
      {}
      &
      k
      &
      {}
    \end{tikzcd}
  \]
\end{definition}

\begin{definition}
  A graded map~$\omega \colon C \to C$ of a graded coalgebra is a \defemph{coderivation} if
  \begin{gather*}
    \Delta \circ \omega
    =
    (\omega \tensor {\id} + {\id} \tensor \omega) \circ \Delta \,;
  \shortintertext{more explicitely,}
    \Delta(\omega(c))
    =
    \sum_{(c)}
    \omega(c_{(1)}) \tensor c_{(2)}
    + (-1)^{\hdeg{\omega} \hdeg{c_{(1)}}} c_{(1)} \tensor \omega(c_{(2)})  \,.
  \end{gather*}
\end{definition}

\begin{remark}
  \leavevmode
  \begin{enumerate}
    \item
      A {\dgc} is the same as a graded coalgebra~$C$ together with a differential~$d$ such that~$d$ vanishes on~$\boundaries_0(C)$ and
      \[
        \Delta(d(c))
        =
        \sum_{(c)}
        d(c_{(1)}) \tensor c_{(2)}
        + (-1)^{\hdeg{c_{(1)}}} c_{(1)} \tensor d(c_{(2)}) \,,
      \]
      i.e.\ such that~$d$ is a graded coderivation of degree~$-1$.
    \item
      The graded cocommutativity of~$C$ means
      \[
        \sum_{(c)} c_{(1)} \tensor c_{(2)}
        =
        \sum_{(c)} (-1)^{\hdeg{c_{(1)}} \hdeg{c_{(2)}}} c_{(2)} \tensor c_{(1)} \,.
      \]
    \item
      A morphism of {\dgcs} is the same as a morphism of the underlying graded coalgebras that commutes with the differentials.
    \item
      Every coalgebra~$C$ is a {\dgc} centered in degree~$0$, in particular~$C = k$.
  \end{enumerate}
\end{remark}

% \begin{example}
%   Let~$V$ be a {\dgv}.
%   Then the induced {\dgv}~$\dgTensor(V)$ becomes a {\dgc} with the deconcatination
%   \begin{align*}
%     \Delta
%     \colon
%     \dgTensor(V)
%     \to
%     \dgTensor(V) \tensor \dgTensor(V) \,,
%     \quad
%     &v_1 \dotsm v_n
%     \mapsto
%     \sum_{i=0}^n
%     v_1 \dotsm v_i \tensor v_{i+1} \dotsm v_n \,,
%   \\
%     \varepsilon
%     \colon
%     \dgTensor(V)
%     \to
%     k \,,
%     \quad
%     &v_1 \dotsm v_n
%     \mapsto
%     \delta_{n0} \,.
%   \end{align*}
% %   This {\dgc} is in general not graded cocommutative because
% %   \begin{align*}
% %     \Delta(vw)
% %     &=
% %     vw \tensor 1 + v \tensor w + 1 \tensor vw \,,
% %   \\
% %     \tau \circ \Delta(vw)
% %     &=
% %     1 \tensor vw + (-1)^{\hdeg{v}\hdeg{w}} w \tensor v + vw \tensor 1 \,.
% %   \end{align*}
% \end{example}

\begin{lemma}
  Let~$C$,~$D$ be {\dgcs}.
  \begin{enumerate}
    \item
      The tensor product~$C \tensor D$ becomes a {\dgc} with
      \begin{gather*}
        \Delta_{C \tensor D}
        \colon
        C \tensor D
        \xlongto{\Delta \tensor \Delta}
        C \tensor C \tensor D \tensor D
        \xlongto{{\id} \tensor \tau \tensor {\id}}
        C \tensor D \tensor C \tensor D
        \\
        \varepsilon_{C \tensor D}
        \colon
        C \tensor D
        \xlongto{\varepsilon \tensor \varepsilon}
        k \tensor k
        \xlongto{\sim}
        k
      \end{gather*}
    \item
      If~$f \colon C \to C'$ and~$g \colon D \to D'$ are morphism of {\dgcs} then so is~$f \tensor g$.
    \item
      The twist map~$\tau \colon C \tensor D \to D \tensor C$ is a morphism of {\dgcs}.
    \item
      If~$C = (C, \Delta, \varepsilon)$ then~$C^{\cop} = (C, \Delta^{\cop}, \varepsilon)$ with~$\Delta^{\op} = \tau \circ \Delta$ is again a {\dgc}.
  \end{enumerate}
\end{lemma}

\begin{warning}
  If~$C \tensor_k D$ is the non-dg tensor product then~$C \tensor D \neq C \tensor_k D$ as coalgebras, i.e.\ the underlying coalgebra of~$C \tensor D$ is not the tensor product of the underlying coalgebras of~$C$ and~$D$.
  The underlying coalgebra of~$C^{\op}$ is similarly not the coopposite of the underlying coalgebra of~$C$.
\end{warning}

\begin{definition}
  A \defemph{\dgci} in a {\dgc}~$C$ is a {\dgsub} that is a coideal.
\end{definition}

\begin{lemma}
  If~$I$ is a {\dgci} in~$C$ then~$C/I$ inherits a~{\dgc} structure.
  \qed
\end{lemma}

\begin{proposition}
  If~$C$ is a {\dgc} then~$\cycles(C)$ is a graded subcoalgebra of~$A$,~$\boundaries(C)$ is a graded coideal in~$\cycles(C)$ and~$\homology(C)$ is hence a graded coalgebra.
  \qed
\end{proposition}






\section{Differential Graded Bialgebras}

\begin{lemma}
  \label{characterization of bialgebras}
  Let~$B$ be a {\dgv},~$(B, m, u)$ a {\dga} and~$(B, \Delta, \varepsilon)$ a {\dgc}.
  Then the following are equivalent:
  \begin{enumerate}
    \item
      $\Delta$ and~$\varepsilon$ are morphisms of {\dgas}.
    \item
      $m$ and~$u$ are morphisms of {\dgcs}.
    \qed
  \end{enumerate}
\end{lemma}

% \begin{proof}
%   The same diagramatic proof as in the non-dg case.
% \end{proof}

\begin{definition}
  A \defemph{\dgb} is a quintuple~$(B, \mu, u, \Delta, \varepsilon)$ such that the equivalent conditions of \cref{characterization of bialgebras} are satisfied.
  A map~$f \colon B \to C$ is a \defemph{morphism} of {\dgbs} if it is both a morphism of {\dgas} and of {\dgcs}.
  A \defemph{\dgbi} is a {\dgsub} that is both a {\dgi} and a {\dgci}.
\end{definition}

\begin{remark}
  The compatibility of the multiplication and comultiplication of~$B$ means
  \begin{gather*}
    \Delta(bc)
%     =
%     \Delta(b)\Delta(c)
%     =
%     \left( \sum_{(b)} b_{(1)} \tensor b_{(2)} \right)
%     \left( \sum_{(c)} c_{(1)} \tensor c_{(2)} \right)
    =
    \sum_{(b), (c)}
    (-1)^{\hdeg{b_{(2)}} \hdeg{c_{(1)}}}
    b_{(1)} c_{(1)} \tensor b_{(2)} c_{(2)}
  \end{gather*}

\end{remark}


\begin{warning}
  A {\dgb} does in general \emph{not} have an underlying bialgebra structure:
  The comultiplication~$\Delta \colon B \to B \tensor B$ is a morphism of {\dgas} where the algebra structure on~$B \tensor B$ is given by~$(b \tensor b') \cdot (b'' \tensor b''') = (-1)^{\hdeg{b'} \hdeg{b''}} b b'' \tensor b' b'''$.
  But it is in general not an algebra homomorphism with respect to the multiplication~$(b \tensor b') \cdot (b'' \tensor b''') = b b'' \tensor b' b'''$.
  \[
    \begin{tikzcd}[column sep = small]
      \text{algebras}
      &
      {}
      &
      \text{\dgas}
      \arrow[dashed]{ll}
      &
      {}
      \\
      {}
      &
      \text{bialgebras}
      \arrow{ul}
      \arrow{dl}
      &
      {}
      &
      \text{\dgbs}
      \arrow{ul}
      \arrow[dashed]{ll}[{anchor=center,sloped}]{\bigg/}
      \arrow{dl}
      \\
      \text{coalgebras}
      &
      {}
      &
      \text{\dgcs}
      \arrow[dashed]{ll}
      &
      {}
    \end{tikzcd}
  \]
  We will see an explicit counterexample in \cref{exterior hopf algebra}.
\end{warning}

% \begin{lemma}
%   If~$B$ is a {\dgb} then~$B^{\op}$,~$B^{\cop}$ and~$B^{\op,\cop}$ are again {\dgbs}.
%   \qed
% \end{lemma}

\begin{lemma}
  If~$I$ is a {\dgbi} in~$B$ then~$B/I$ inherits a {\dgb} structure.
  \qed
\end{lemma}

\begin{proposition}
  If~$B$ is a {\dgb} then~$\cycles(B)$ is a graded sub-bialgebra of~$B$,~$\boundaries(B)$ is a graded biideal in~$\cycles(B)$ and~$\homology(B)$ is hence a graded bialgebra.
  \qed
\end{proposition}

\begin{definition}
  If~$B$ is a {\dgb} then~$x \in B$ is \defemph{primitive} if~$\Delta(x) = x \tensor 1 + 1 \tensor x$.
\end{definition}

\begin{lemma}
  If~$x, y \in B$ are primitive then~$[x,y]$ is again primitive.
  \qed
\end{lemma}






\section{Differential Graded Hopf Algebras}

\begin{lemma}
  If~$C$ is a {\dgc} and~$A$ is a {\dga} then the convolution product on~$\Hom_k(C,A)$ makes~$\dgHom(C,A)$ into a {\dga}.
  \qed
\end{lemma}

% TODO: Dual of dg-coalgebra is a dg-algebra.

\begin{definition}
  An \defemph{antipode} for a~{\dgb}~$H$ is an inverse~$S$ to~$\id_H$ with respect to the convolution product of~$\dgHom(H,H)$.
  If~$H$ admits an antipode then it is a \defemph{\dgh}.
  A \defemph{morphism} of {\dghs} is a morphism of {\dgbs}.
  A \defemph{\dghi} in~$H$ is a {\dgbi}~$I$ with~$S(I) \subseteq I$.
\end{definition}

\begin{warning}
  A {\dgh} need not have an underlying Hopf algebra structure.
\end{warning}

\begin{remark}
  The antipode of a {\dgh}~$H$ is the the unique morphism of {\dgvs}~$S \colon H \to H$ that makes the diagram
  \begin{equation}
    \label{antipode diagram}
    \begin{tikzcd}[column sep = small]
      {}
      &
      H \tensor H
      \arrow{rr}[above]{S \tensor {\id}}
      &
      {}
      &
      H \tensor H
      \arrow{dr}[above right]{m}
      &
      {}
      \\
      H
      \arrow{ur}[above left]{\Delta}
      \arrow{rr}[above]{\varepsilon}
      \arrow{dr}[below left]{\Delta}
      &
      {}
      &
      k
      \arrow{rr}[above]{u}
      &
      {}
      &
      H
      \\
      {}
      &
      H \tensor H
      \arrow{rr}[below]{{\id} \tensor S}
      &
      {}
      &
      H \tensor H
      \arrow{ur}[below right]{m}
      &
      {}
    \end{tikzcd}
  \end{equation}
  commute.
  This means more explicitely that
  \[
    \sum_{(c)} S(c_{(1)}) c_{(2)}
    =
    \varepsilon(c) 1_H
    \qquad\text{and}\qquad
    \sum_{(c)} c_{(1)} S(c_{(2)})
    =
    \varepsilon(c) 1_H  \,.
  \]
  (No additional signs occur because~$\hdeg{S} = 0$.)
\end{remark}

\begin{lemma}
  If~$I$ is a {\dghi} in~$H$ then~$H/I$ a {\dgh} structure.
  \qed
\end{lemma}

% \begin{lemma}
%   Let~$H$ be a~{\dgb} and let~$S \colon H \to H^\op$ be a morphism of~{\dgas}.
%   If the diagram~\eqref{antipode diagram} commutes on an algebra generating set of~$H$ then~$S$ is an antipode for~$H$.
% \end{lemma}
% 
% \begin{proof}
%   The sets
%   \[
%     \{
%       x \in H
%     \suchthat
%       m \circ (S \tensor {\id}) \circ \Delta (x)
%       =
%       \varepsilon(x) 1_H
%     \}
%   \]
%   and
%   \[
%     \{
%      x \in H
%     \suchthat
%       m \circ ({\id} \tensor S) \circ \Delta (x)
%       =
%       \varepsilon(x) 1_H
%     \}
%   \]
%   subalgebras of~$H$ that contain all algebra generators.
% \end{proof}


\begin{example}
  Let~$V$ be a {\dgv}.
  \begin{enumerate}
    \item
      The map
      \[
        V
        \to
        \dgTensor(V) \tensor \dgTensor(V) \,,
        \quad
        v
        \mapsto
        v \tensor 1 + 1 \tensor v
      \]
      is a morphism of {\dgvs} and hence induces a morphism of~{\dgas}
      \[
        \Delta
        \colon
        \dgTensor(V)
        \to
        \dgTensor(V) \tensor \dgTensor(V) \,.
      \]
      The zero map~$V \to 0$ induces a morphism of~{\dgas}
      \[
        \varepsilon
        \colon
        \dgTensor(V)
        \to
        \dgTensor(0)
        =
        k \,.
      \]
      These maps make~$\dgTensor(V)$ into a {\dgb};
      the necessary diagrams can be checked on the algebra generators~$V$ of~$\dgTensor(V)$ because all arrows occuring in the bialgebra diagrams are morphisms of {\dgas}.
      The maps~$\Delta$ and~$\varepsilon$ are explicitely given by
      \begin{align*}
        \Delta(v_1 \dotsm v_n)
        &=
        \Delta(v_1) \dotsm \Delta(v_n)
        \\
        &=
        (v_1 \tensor 1 + 1 \tensor v_1)
        \dotsm
        (v_n \tensor 1 + 1 \tensor v_n)
        \\
        &=
        \sum_{p=0}^n
        \;
        \sum_{\sigma \in \Sh(p,n-p)}
        (-1)^{n_p(\sigma)}
        v_{\sigma(1)} \dotsm v_{\sigma(p)}
        \tensor
        v_{\sigma(p+1)} \dotsm v_{\sigma(n)}
      \end{align*}
      where
      \[
        n_p(\sigma)
        =
        \sum
        \Bigl\{
          \hdeg{v_i} \hdeg{v_j}
        \suchthat[\Big]
          1 \leq i \leq p, \;
          p+1 \leq j \leq n, \;
          \sigma(i) > \sigma(j)
        \Bigr\} \,,
      \]
      and
      \[
        \varepsilon( v_1 \dotsm v_n )
        =
        \begin{cases}
          1 & \text{if~$n = 0$}, \\
          0 & \text{otherwise}.
        \end{cases}
      \]
      The map
      \[
        V
        \to
        \dgTensor(V)^{\op} \,,
        \quad
        v
        \mapsto
        -v
      \]
      is a morphism of~{\dgvs} and hence induces a morphism of {\dgas}
      \[
        S
        \colon
        \dgTensor(V)
        \to
        \dgTensor(V)^{\op} \,.
      \]
      As a map~$S \colon \dgTensor(V) \to \dgTensor(V)$ this is given by
      \[
        S(v_1 \dotsm v_n)
        =
        (-1)^{\sum_{1 \leq i < j \leq n} \hdeg{v_i} \hdeg{v_j}}
        (-1)^n
        v_n \dotsm v_1  \,.
      \]%
%     TODO: Find an argument to check this only on algebra generators.
      It can now be checked on the monomials~$v_1 \dotsm v_n$ that~$S$ is an antipode for~$\dgTensor(V)$, making it a {\dgh}.
%       \footnote{In the resulting expressions the terms for~$v_1 \dotsm v_p \tensor v_{p+1} \dotsm v_n$ and~$v_2 \dotsm v_p \tensor v_1 v_{p+1} \dotsm v_n$ cancel out because of signs.}
    \item
      The {\dga}~$\dgSymm(V) = \dgTensor(V)/I$ from \cref{dg symmetric algebra} inherits from~$\dgTensor(V)$ the structure of a {\dgh} because the~{\dgi}
      \[
        I
        =
        \bigl(
          [v,w]
        \suchthat[\big]
          \text{$v,w \in V$ are homogeneous}
        \bigr)
      \]
      is a {\dghi} in~$\dgTensor(V)$, since
      \begin{align*}
        \varepsilon([v,w])
        &=
        0 \,,
      \\
        \Delta([v,w])
        &=
        [v,w] \tensor 1 + 1 \tensor [v,w]
        \in
        I \tensor \dgTensor(V) + \dgTensor(V) \tensor I \,,
      \\
        S([v,w])
        &=
        -[v,w]
        \in
        I \,.
      \end{align*}
      For the computation of~$\Delta$ we use that~$v$,~$w$ are primitive in~$\dgTensor(V)$ and~$[v,w]$ is therefore again primitive.
  \end{enumerate}
\end{example}

\begin{example}[Exterior Algebra]
  \label{exterior hopf algebra}
  Let~$V$ be a vector space.
  We regard~$V$ as a {\dgv} concentrated in degree~$1$.
  Then~$\dgSymm(V) = \Exterior(V)$ as graded algebras whence~$\Exterior(V)$ is a graded Hopf algebra.
  But for~$\ringchar k \neq 2$ there exists no bialgebra structure on~$\Lambda \defined \Exterior(V)$;
  we prove this in \cref{exterior hopf algebra proof}.
\end{example}

\begin{proposition}
  If~$\hopf$ is a {\dgh} with antipode~$S$ then the graded bialgebra~$\homology(\hopf)$ is a graded Hopf algebra with antipode induced by~$S$.
  \qed
\end{proposition}

\begin{example}
  If~$V$ is a {\dgv} then
  \[
    \homology(\dgTensor(V))
    =
    \homology\Biggl( \bigoplus_{d \geq 0} V^{\tensor d} \Biggr)
    \cong
    \bigoplus_{d \geq 0} \homology\bigl( V^{\tensor d} \bigr)
    \cong
    \bigoplus_{d \geq 0} \homology(V)^{\tensor d}
    =
    \dgTensor(\homology(V))
  \]
  as graded vector spaces by the algebraic Künneth isomorphism.
  We see on representatives that this is already an isomorphism of graded Hopf algebras.
\end{example}

% TODO: Give the isomorphism on representatives.

% TODO: Do this for the graded symmetric algebra by using the symmetrization operator.





% \section{Chevalley--Eilenberg}
% 
% For this section we fix a Lie~algebra~$\glie$.
% The algebra morphism~$\varepsilon \colon \Univ(\glie) \to k = \End_k(k)$ makes the ground field~$k$ into a symmetric~{\bimodule{$\Univ(\glie)$}}.
% 
% 
% 
% 
% 
% \subsection{The Chevalley--Eilenberg Complex}
% 
% \begin{definition}
%   The \defemph{{\CE} complex} of~$\glie$ is in degree~$n$ given by~$\Univ(\glie) \tensor \Exterior^n(\glie)$ and the differential~$\dce$ is given by
%   \begin{align*}
%     {}&
%     \dce(u \tensor x_1 \wedge \dotsb \wedge x_n)
%     \\
%     ={}&
%     \sum_{i=1}^n (-1)^i u x_i \tensor x_1 \wedge \dotsb \wedge \widehat{x_i} \wedge \dotsb \wedge x_n
%     \\
%     {}&
%     +
%     \sum_{1 \leq i < j \leq n}
%     (-1)^{i+j-1}
%     u \tensor [x_i, x_j] \wedge x_1 \wedge \dotsb \wedge \widehat{x_i} \wedge \dotsb \wedge \widehat{x_j} \wedge \dotsb \wedge x_n \,.
%   \end{align*}
% \end{definition}
% 
% \begin{remark}
%   If we set~$[u, x] \defined u x$ and~$(x_0, x_1, \dotsc, x_n) \defined x_0 \tensor x_1 \wedge \dotsb \wedge x_n$ then
%   \[
%     \dce(x_0, x_1, \dotsc, x_n)
%     =
%     \sum_{0 \leq i < j \leq n}
%     (-1)^{j+1}
%     (x_0, x_1, \dotsc, [x_i, x_j], \dotsc, \widehat{x_j}, \dotsc, x_n)
%   \]
%   where~$[x_i, x_j]$ appears in the~{\howmanyth{$i$}} position.
% \end{remark}
% 
% % A representation of~$\glie$ is a vector space~$V$ together with homomorphism of Lie~algebras~$\rho \colon \glie \to \End_k(V)$, or equivalently a bilinear action~$\glie \times V \to V$,~$(x,v) \mapsto x v$ with
% % \[
% %   x (y v) - y (x v)
% %   =
% %   [x,y] v \,.
% % \]
% % It follows from the correspondence
% % \begin{align*}
% %   {}&
% %   \{
% %     \text{Lie algebra homomorphisms~$\glie \to \End_k(V)$}
% %   \}
% %   \\
% %   \longonetoone{}& 
% %   \{
% %     \text{algebra homomorphism~$\Univ(\glie) \to \End_k(V)$}
% %   \}
% % \end{align*}
% % that a representation of~$\glie$ is the same as a~{$\Univ(\glie)$}{module}.
% % The action of~$\Univ(\glie)$ extends the corresponding action of~$\glie$.
% 
% \begin{theorem}
%   The {\CE} complex together with the counit~$\varepsilon \colon \Univ(\glie) \to k$ is a projective resolutions of~$k$ as a left~{\module{$\Univ(\glie)$}}.
% \end{theorem}
% 
% \begin{proof}
%   A proof due to Koszul using spectrac sequences can be found in \cite[Theorem~7.7.2]{weibel}.
%   A more elementary proof can be found in \cite[VII.4]{hilton_stammbach}.
% \end{proof}
% 
% 
% \begin{remark}
%   The Lie~algebra cohomology of~$\glie$ with values in a left~{\module{$\Univ(\glie)$}}~$V$ is given by
%   \begin{align*}
%     \liecohom(\glie, V)
%     &\defined
%     \Ext_{\Univ(\glie)}(k, V)
%   \intertext{and the Lie~algebra homology of~$\glie$ with values in a a right~{\module{$\Univ(\glie)$}}~$V$ is given by}
%     \liehom(\glie, V)
%     &\defined
%     \Tor^{\Univ(\glie)}(V,k) \,.
%   \end{align*}
%   The {\CE} complex can be used to compute these:
%   The Lie~algebra cohomology of~$\glie$ is the cohomology of the cochain complex
%   \[
%     \Hom_{\Univ(\glie)}\biggl( \Univ(\glie) \tensor \Exterior(\glie), V \biggr)
%     \cong
%     \Hom_k\biggl( \Exterior(\glie), V \biggr)
%   \]
%   that is in degree~$n$ given by
%   \[
%     \Hom_k\biggl( \Exterior^n(\glie), V \biggr)
%     \cong
%     \{
%       \text{alternating multilinear maps~$\glie^{\times n} \to V$}
%     \}
%   \]
%   and has the differential
%   \begin{align*}
%     d(\omega)(x_1, \dotsc, x_n)
%     ={}&
%     \sum_{i=1}^n
%     (-1)^i x_i \omega(x_1, \dotsc, \widehat{x_i}, \dotsc, x_n)
%     \\
%     {}&
%     + \sum_{1 \leq i < j \leq n}
%       (-1)^{i+j-1} \omega([x_i, x_j], x_1, \dotsc, \widehat{x_i}, \dotsc, \widehat{x_j}, \dotsc, x_n) \,.
%   \end{align*}
%   The Lie~algebra homology of~$\glie$ is the homology of the chain complex
%   \[
%     V \tensor_{\Univ(\glie)} \Univ(\glie) \tensor_k \Exterior(\glie)
%     \cong
%     V \tensor_k \Exterior(\glie)
%   \]
%   that has the differential
%   \begin{align*}
%     {}&
%     d(v \tensor x_1 \wedge \dotsb \wedge x_n)
%     \\
%     ={}&
%     \sum_{i=1}^n (-1)^i (v \cdot x_i) \tensor x_1 \wedge \dotsb \wedge \widehat{x_i} \wedge \dotsb \wedge x_n
%     \\
%     {}&
%     +
%     \sum_{1 \leq i < j \leq n}
%     (-1)^{i+j-1}
%     v \tensor [x_i, x_j] \wedge x_1 \wedge \dotsb \wedge \widehat{x_i} \wedge \dotsb \wedge \widehat{x_j} \wedge \dotsb \wedge x_n \,.
%   \end{align*}
% \end{remark}
% 
% 
% 
% \subsection{The Chevalley--Eilenberg Coalgebra}
% 
% For~$V = \glie$ we get a chain complex~$\Exterior \glie$ with Chevalley--Eilenberg differential
% \[
%   \dce(x_1 \wedge \dotsb \wedge x_n)
%   =
%   \sum_{1 \leq i < j \leq n}
%   (-1)^{i+j-1}
%   [x_i, x_j]
%   \wedge x_1
%   \wedge \dotsb
%   \wedge \widehat{x_i}
%   \wedge \dotsb
%   \wedge \widehat{x_j}
%   \wedge \dotsb
%   \wedge x_n \,.
% \]
% We observe that the differential~$\Exterior^2 \glie \to \glie$ is precisely the Lie~bracket~$[-,-]$.
% The composition~$\Exterior^3 \glie \to \Exterior^2 \glie \to \glie$ is given by
% \begin{align*}
%   x \wedge y \wedge z
%   &\mapsto
%   [x,y] \wedge z - [x,z] \wedge y + [y,z] \wedge x
%   \\
%   &\mapsto
%   [[x,y], z] - [[x,z], y] + [[y,z], x]
%   \\
%   &=
%   -\Bigl( [z,[x,y]] + [y,[z,x]] + [x,[y,z]] \Bigr)
% \end{align*}
% We can regard~$\Exterior \glie$ as a graded coalgebra as in \cref{exterior hopf algebra}, i.e.\ such that~$V$ consists of primitive elements.
% 
% \begin{proposition}
%   The Chevalley--Eilenberg differential makes~$\Exterior \glie$ into a {\dgc}, and it is the unique extension of the Lie~bracket to a coderivation of~$\Exterior \glie$.
%   Any coderivation on~$\Exterior \glie$ comes from a Lie~algebra structure of~$\glie$.
%   This gives a one-to-one correspondence between Lie~brackets on~$\glie$ and coderivations on~$\bigwedge \glie$.
%   \todo{Find a proper reference.}
% \end{proposition}
% 
% 
% 
% 
% 
% \subsection{The Chevalley--Eilenberg Algebra}%
% 
% 
% \todo{Write this, check duality with CE-coalgebra.}





\section{Differential Graded Lie Algebras}

Let~$\ringchar(k) \neq 2$.

\begin{recall}
  A Lie~algebra is a vector space~$\glie$ together with a map~$[-,-] \colon \glie \tensor_k \glie \to \glie$ such that~$[-,-]$ is skew-symmetric and for every~$x \in \glie$ the map~$[x,-] \colon \glie \to \glie$ is a derivation;
  the last assertion is equivalent to the Jacobi identity~$\sum_{\text{cyclic}} [x,[y,z]] = 0$.
\end{recall}

\begin{definition}
  A \defemph{\dgl} is a {\dgv}~$\glie$ together with a morphism~$[-,-] \colon \glie \tensor \glie \to \glie$ such that~$[-,-]$ is \defemph{graded skew symmmetric}, i.e.\ such that the diagram
  \[
    \begin{tikzcd}[column sep = small]
      \glie \tensor \glie
      \arrow{rr}[above]{\tau}
      \arrow{dr}[below left]{[-,-]}
      &
      {}
      &
      \glie \tensor \glie
      \arrow{dl}[below right]{-[-,-]}
      \\
      {}
      &
      \glie
      &
      {}
    \end{tikzcd}
  \]
  commutes, and such that~$[x,-]$ is for every~$x$ a derivation of degree~$\hdeg{x}$.
\end{definition}

\begin{remark}
  Let~$\glie$ be a~{\dgl}.
  Then~$[\glie_i, \glie_j] \subseteq \glie_{i+j}$ for all~$i,j$ and
  \begin{align}
    [x,y]
    &=
    -(-1)^{\hdeg{x} \hdeg{y}} [y,x]
    \notag
  \shortintertext{and}
    [x, [y,z]]
    &=
    [[x,y], z]
    +
    (-1)^{\hdeg{x} \hdeg{y}}
    [y, [x,z]]
    \label{pre dgjacobi}
  \shortintertext{and}
    d([x,y])
    &=
    [d(x), y] + (-1)^{\hdeg{x}} [x, d(y)] \,.
    \notag
  \end{align}
  We can rewrite~\eqref{pre dgjacobi} as the \defemph{graded Jacobi identity}
  \[
    \sum_{\text{cyclic}}
    (-1)^{\hdeg{x} \hdeg{z}} [x, [y,z]]
    =
    0 \,.
  \]
\end{remark}

\begin{warning}
  A {\dgl} need not have an underlying Lie~algebra structure.
\end{warning}

\begin{example}
  \leavevmode
  \begin{enumerate}
    \item
      Every {\dga}~$A$ becomes a {\dgl} with the {\dgcom}.
    \item
      For a graded algebra~$A$ then the graded subspace~$\dgDer(A) \subseteq \dgEnd(A)$ given by
      \[
        \dgDer(A)_n
        \defined
        \{
          \text{derivations of~$A$ of degree~$n$}
        \}
        \subseteq
        \dgEnd(A)_n
      \]
      is a {\dglsub} of~$\dgEnd(A)$.
    \item
      In any {\dgb}~$B$ the subspace of primitive elements
      \[
        \prim(B)
        =
        \{
          x \in B
        \suchthat
          \Delta(x) = x \tensor 1 + 1 \tensor x
        \}
      \]
      is a {\dglsub} of~$B$.
  \end{enumerate}
\end{example}

\begin{lemma}
  If~$\glie$ is a {\dgl} then~$\cycles(\glie)$ is a graded Lie~subalgebra of~$\glie$,~$\boundaries(\glie)$ is a graded Lie~ideal in~$\cycles(\glie)$ and~$\homology(\glie)$ is thus an graded Lie~algebra. 
\end{lemma}


\begin{definition}
  The \defemph{universal enveloping algebra} of a {\dgl}~$\glie$ is a {\dga}~$\Univ(\glie)$ together with a morphism of~{\dgls}~$i \colon \glie \to \Univ(\glie)$ such that for every other {\dga}~$A$ and every morphism of~{\dgls}~$f \colon \glie \to A$ there exists a unique morphism of {\dgas}~$F \colon \Univ(\glie) \to A$ that extends~$f$:
  \[
    \begin{tikzcd}
      \Univ(\glie)
      \arrow[dashed]{r}[above]{F}
      &
      A
      \\
      \glie
      \arrow{u}[left]{i}
      \arrow{ur}[below right]{f}
      &
      {}
    \end{tikzcd}
  \]
\end{definition}

\begin{proposition}
  Every {\dgl}~$\glie$ admits a universal enveloping algebra.
  It is unique up to unique isomorphism and can be constructed as
  \[
    \Univ(\glie)
    =
    \dgTensor(\glie)
    /
    \bigl(
      [x,y]_{\dgTensor(\glie)} - [x,y]_{\glie}
    \suchthat
      \text{$x, y \in \glie$ homogeneous}
    \bigr)
  \]
  together with the composition~$i \colon \glie \to \dgTensor(\glie) \to \Univ(\glie)$.
  It inherits from~$\dgTensor(\glie)$ the structure of a {\dgh}.
  \qed
\end{proposition}

\begin{proof}
  We check that the given ideal~$I$ is a {\dghi}.
  It is generated by homegenous elements which satisfy
  \begin{align*}
    {}&
    d([x,y]_{\dgTensor(\glie)} - [x,y]_{\glie})
    \\
    ={}&
    d([x,y]_{\dgTensor(\glie)}) - d([x,y]_{\glie})
    \\
    ={}&
      [d(x), y]_{\dgTensor(\glie)}
    + (-1)^{\hdeg{x}} [x, d(y)]_{\dgTensor(\glie)}
    - [d(x), y]_{\glie}
    - (-1)^{\hdeg{x}} [x, d(y)]_{\glie}
    \\
    ={}&
    \biggl(
      [d(x), y]_{\dgTensor(\glie)} - [d(x), y]_{\glie}
    \biggr)
    + 
    (-1)^{\hdeg{x}}
    \biggl(
      [x, d(y)]_{\dgTensor(\glie)} - [x, d(y)]_{\glie}
    \biggr)
    \\
    \in{}&
    I
  \end{align*}
  so it is a {\dgi}.
  Also
  \[
    \varepsilon([x,y]_{\dgTensor(\glie)} - [x,y]_{\glie})
    =
    \varepsilon([x,y]_{\dgTensor(\glie)}) - \varepsilon([x,y]_{\glie})
    =
    0 - 0
    =
    0
  \]
  because~$[x,y]_{\Tensor(\glie)}$ and~$[x,y]_{\glie}$ are homogoneous of degree~$\geq 1$,
  \begin{gather*}
    \begin{aligned}
      {}&
      \Delta([x,y]_{\dgTensor(\glie)} - [x,y]_{\glie})
      \\
      ={}&
      \Delta([x,y]_{\dgTensor(\glie)}) - \Delta([x,y]_{\glie}))
      \\
      ={}&
        [x,y]_{\dgTensor(\glie)} \tensor 1
      + 1 \tensor [x,y]_{\dgTensor(\glie)}
      - [x,y]_{\glie} \tensor 1
      - 1 \tensor [x,y]_{\glie}
      \\
      ={}&
        ([x,y]_{\dgTensor(\glie)} - [x,y]_{\glie}) \tensor 1
      + 1 \tensor ([x,y]_{\dgTensor(\glie)} - [x,y]_{\glie})
      \\
      \in{}&
      I \tensor \dgTensor(\glie) + \dgTensor(\glie) \tensor I
    \end{aligned}
  \intertext{since both~$[x,y]_{\dgTensor(\glie)}$ and~$[x,y]_{\glie}$ are primitive, and finally}
    S([x,y]_{\dgTensor(\glie)} - [x,y]_{\glie})
    =
    S([x,y]_{\dgTensor(\glie)}) - S([x,y]_{\glie})
    =
    -[x,y]_{\dgTensor(\glie)}  + [x,y]_{\glie}
    \in
    I \,.
  \end{gather*}
  Thus the {\dgi}~$I$ is already a~{\dghi}.
\end{proof}


\begin{remark}
  Let~$\glie$,~$\hlie$ be a {\dgls}.
  \begin{enumerate}
%     \item
%       The product~$\glie \times \hlie$ is again a {\dgl} with
%       \[
%         [(x,y), (x',y')]
%         =
%         ( [x,x'], [y,y'] )  \,.
%       \]
%       The inclusions~$\glie, \hlie \to \glie \times \hlie$ induce morphisms of {\dghs}
%       \begin{align*}
%         \Univ(\glie), \Univ(\hlie)
%         \to&
%         \Univ(\glie \times \hlie)
%       \intertext{that results in an isomorphism of {\dghs}}
%         \Univ(\glie) \tensor \Univ(\hlie)
%         \cong&
%         \Univ(\glie \times \hlie) \,.
%       \end{align*}
%     \item
%       The Hopf algebra structure of~$\Univ(\glie)$ is induced from underlying morphisms of {\dgls}:
%       The diagonal morphism~$\glie \to \glie \times \glie$,~$v \mapsto (v,v)$ induces the comultiplication
%       \[
%         \Univ(\glie) 
%         \to
%         \Univ(\glie \times \glie)
%         \cong
%         \Univ(\glie) \tensor \Univ(\glie)
%       \]
%       the morphism~$\glie \to 0$ induced the counit
%       \[
%         \Univ(\glie)
%         \to
%         \Univ(0)
%         =
%         k
%       \]
%       and the morphism~$\glie \to \glie^{\op}$,~$v \mapsto -v$ induces the antipode
%       \[
%         \Univ(\glie)
%         \to
%         \Univ(\glie^{\op})
%         =
%         \Univ(\glie)^{\op}
%       \]
    \item
      The famous Poincaré–Birkhoff–Witt theorem generalizes to the universal enveloping algebras of {\dgls}.
      It can be expressed as an isomorphism of {\dgc}~$\dgSymm(\glie) \cong \Univ(\glie)$ and shows that~$\prim(\Univ(\glie)) = \glie$.
      Details on this can be found in \cite[Appendix~B,Theorem~2.3]{quillen} and \cite[\S21(a)]{rational_homotopy_book}.
    \item
      It holds that~$\homology(\Univ(\glie)) \cong \Univ(\homology(\glie))$, see \cite[Appendix~B,Proposition~2.1]{quillen} or \cite[Theorem 21.7]{rational_homotopy_book}.
    \item
      If~$H$ is a graded cocommutative connected%
      \footnote{The connectedness is defined in terms of the underlying {\dgc}, not that of the {\dga}.}
      {\dgh} then a version of the Cartier--Milnor--Moore theorem asserts that~$H \cong \Univ(\prim(H))$, which results in an equivalence between the categories of~{\dgls} and cocommutative connected {\dghs}, see \cite[Appendix~B,Theorem~4.5]{quillen}.
  \end{enumerate}
\end{remark}





% \section{Homology of the Primitive Part}
% 
% \begin{theorem}[{\cite[Theorem~A.9]{loday}}]
%   Let~$\hopf$ be a {\dgh}.
%   The inclusion~$\prim(\hopf) \to \hopf$ is a morphism of {\dgls} and thus induced a morphism of graded Lie~algebras~$\homology(\prim(\hopf)) \to \homology(\hopf)$.
%   This morphism restricts to an isomorphism of graded Lie~algebras~$\homology(\prim(\hopf)) \to \prim(\homology(\hopf))$.
% \end{theorem}
% 
% % TODO: Find a proof.





\appendix




\section{Calculations and Proofs}





\subsection{\cref{exterior hopf algebra}}
\label{exterior hopf algebra proof}

Suppose that there exists a bialgebra structure on~$\Exterior(V)$.
Then~$\varepsilon(v)^2 = \varepsilon(v^2) = 0$ and thus~$\varepsilon(v) = 0$ for all~$v \in V$, so~$\ker \varepsilon = \bigoplus_{d \geq 1} \Exterior^n(V) \defines I$.
Let~$v \in V$.
Then by the counital axiom,
\[
  \Delta(v)
  \equiv
  v \tensor 1
  \pmod{\Lambda \tensor I}
  \qquad\text{and}\qquad
  \Delta(v)
  \equiv
  1 \tensor v
  \pmod{I \tensor \Lambda}
\]
and thus
\[
  \Delta(v)
  \equiv
  v \tensor 1 + 1 \tensor v
  \pmod{I \tensor I}  \,.
\]
It follows that
\begin{gather*}
  \Delta(v^2)
  \equiv
  (v \tensor 1 + 1 \tensor v)^2
  \pmod{ (v \tensor 1)(I \tensor I) + (1 \tensor v)(I \tensor I) + (I \tensor I)^2 } \,,
\intertext{and therefore}
  \Delta(v^2)
  \equiv
  v^2 \tensor 1 + 2 v \tensor v + 1 \tensor v^2
  \pmod{I \tensor I^2 + I^2 \tensor I} \,.
\end{gather*}
Now~$v^2 = 0$, hence
\[
  2 v \tensor v
  \equiv
  0
  \pmod{I \tensor I^2 + I^2 \tensor I}  \,.
\]
But~$2 \neq 0$ and~$v \neq 0$ hence~$2 v \tensor v \neq 0$ while~$v \tensor v \notin I \tensor I^2 + I^2 \tensor I$, a contradiction.
(This proof is taken from \cite{exterior_bialgebra_mo} and partially from \cite[III.{\S}11.3]{bourbaki}).











\printbibliography

\end{document}
